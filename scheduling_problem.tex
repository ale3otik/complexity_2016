\documentclass[10pt]{article}
% Эта строка — комментарий, она не будет показана в выходном файле
\usepackage{ucs}
\usepackage[utf8x]{inputenc} % Включаем поддержку UTF8
\usepackage[russian]{babel}  % Включаем пакет для поддержки русского языка
\usepackage{amsmath}
\usepackage{amssymb}
\usepackage{mathtools}

\hoffset=0mm
\voffset=0mm
\textwidth=170mm        % ширина текста
\oddsidemargin=-0mm   % левое поле 25.4 - 5.4 = 20 мм
\textheight=240mm       % высота текста 297 (A4) - 40
\topmargin=-15.4mm      % верхнее поле (10мм)
\headheight=5mm      % место для колонтитула
\headsep=5mm          % отступ после колонтитула
\footskip=8mm         % отступ до нижнего колонтитула

% \textwidth=180mm    
% \oddsidemargin=-10mm 

\title{Задача об оптимальном рассписании}
\author{Зотов Алексей, 497}
\date{\today}

\begin{document}
\maketitle
\paragraph{\Large {Формулировка задачи\\\\}} 

\indent Имеется множество работ $J$ и множество машин $M$. Также задана функция $p: J \times M \to \mathbb{R}_+$. Значение $p(i,j) = p_{ij}$ означает время выполнения $i$-ой работы на $j$-ой машине. \\
Требуется найти распределние работ по машинам, так чтобы время выполнения всех работ было минимально. Формально, требуется построить функцию $x : J \times M \to \{1,0\}$ такую, что: 
\begin{equation}
    \sum_{j \in M} x_{ij} = 1 , \quad \forall i
\end{equation}
\begin{equation}
    \max_{j \in M} \sum_{i} x_{ij} p_{ij} \to \min 
\end{equation}
\\
\\
\paragraph{\Large{$\mathbf{NP}$ - полнота\\\\}}
\indent \textbf{Теорема.} Задача об оптимальном расписании является $\mathbf{NP}$ - полной.
\\
\\
\paragraph{\Large{Полиномиальное приближение\\\\}}
\indent Найдем приближенное полиномиальное частного случай задачи об оптимальном рассписании, когда мощности всех машин одинаковы, то есть $p_{ij} = p_i$. \\ \\ 
Рассмотрим алгоритм \textbf{LPTR} (Longest Processing Time Rule): \\
\begin{enumerate}
    \item Отсортируем работы в порядке невозрастания сложности работы $p_i$, то есть так, что $p_i \geq p_{i+1} \quad \forall i$. 
    \item Пройдем все работы в данном порядке, назначая на $i$-м шаге данную работу той машине, которая завершит обработку уже назначенных ей задач раньше всех остальных машин.
\end{enumerate}

\indent \textbf{Теорема.} LPTR - имеет коэффициент аппроксимации $\frac{4}{3}$ для задачи об оптимальном расписании, 
то есть $ \text{LPTR} (s) \leq \frac{4}{3} \text{OPT} (s) \quad \forall s$ - входные данные , $\text{OPT} (s)$ - оптимальное решение. 


\end{document}